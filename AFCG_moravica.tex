%%%%%%%%%%%%%%%%%%%%%%%%%%%%%%%%%%%%%%%%%%%%%%%%%%%%%%%%%%%%%%%%%%%%%%
%%%%%%%%%%%%%%% Mathematica Moravica document template %%%%%%%%%%%%%%%
%%%%%%%%%%%%%%%%%%%%%%%%%%%%%%%%%%%%%%%%%%%%%%%%%%%%%%%%%%%%%%%%%%%%%%

\documentclass[b5cutpaper, twoside, 11pt, leqno]{moravica}
\usepackage{amsmath}
\usepackage{amsthm}
\usepackage{amsfonts}
\usepackage{amssymb}
\usepackage{graphicx}
\usepackage[left=1.95cm, top=1.75cm, bottom=1.75cm, right=1.95cm, nofoot]{geometry}
\usepackage[T1]{fontenc}
\usepackage{gensymb}

% allow equation break from page to page
\allowdisplaybreaks

\usepackage[absolute]{textpos}

\usepackage{fancyhdr}
\pagestyle{fancy}
\fancyhead[RO,RE,LO,LE, CO, CE]{}
\fancyfoot{}

\fancyhead[RO]{\footnotesize{\thepage}}
\fancyhead[LE]{\footnotesize{\thepage}}
\fancyhead[CE]{\footnotesize\textsc{\rightmark}}
\fancyhead[CO]{\footnotesize\textsc{\leftmark}}
\fancyfoot{}
\renewcommand{\headrulewidth}{0.5pt}
\renewcommand{\footrulewidth}{0pt}
\footskip=0cm
\headheight=15pt

%%%%%%%%%%%%%%%%%%%%%%%%%%%%%%%%%%%%%%%%%%%%%%%%%%%%%%%%%%%
%%%%%%%%%% Defining Length of the Address box%%%%%%%%%%%%%%
%%%%%%%%%%%%%%%%%%%%%%%%%%%%%%%%%%%%%%%%%%%%%%%%%%%%%%%%%%%
\newlength{\alength}
\setlength{\alength}{10cm}
%%%%%%%%%%%%%%%%%%%%%%%%%%%%%%%%%%%%%%%%%%%%%%%%%%%%%%%%%%%%
%%%%%%%%%%%%%%%% Mathematica Moravica Logo %%%%%%%%%%%%%%%%%
%%%%%%%%%%%%%%%%%%%%%%%%%%%%%%%%%%%%%%%%%%%%%%%%%%%%%%%%%%%%

\newcommand{\nlto}{n < \omega}
\newcommand{\eps}{\epsilon}
\newcommand{\epzero}{\eps > 0}
\newcommand{\gd}{G_{\delta}}

%\def\perf{Perf(\cantor)}

\newcommand{\afcp}{{\mathbf AFC}^\prime}
\newcommand\afcstar{AFC^*_{G}}
\newcommand\afcg{\afcp_{G}}
\newcommand\trans{{\it Trans}(\ca)}
\newcommand\ooo{\underline{O}}
%%%\newcommand{\square}{\hbox{\ \ \ \ \ \vrule\vbox{\hrule\phantom{o}\hrule}\vrule}}
\newcommand{\eee}{{\cal E}}
\newcommand{\seq}{\subseteq}
\newcommand{\ca}{2^{\omega}}
\newcommand{\afcbar}{\overline{AFC}}
\newcommand{\arr}{\rightarrow}
\newcommand{\Arr}{\Rightarrow}
\newcommand{\afc}{AFC}
\newcommand{\mgr}{{\cal M}}
\newcommand{\neglig}{{\cal N}}
\newcommand{\oo}{\omega}
%%%\newcommand{\qed}{\sharp}
%%%\newcommand{\int}{\hbox{\it int}}
\newcommand{\finsub}{[\oo]^{<\oo}}
\newcommand{\infsub}{[\oo]^{\oo}}
\newcommand{\incr}{\oo^{\uparrow \oo }}
\newcommand{\baire}{\oo^{\oo}}
\newcommand{\up}{\uparrow}
\newcommand{\real}{\bf R}
\newcommand{\perf}{{\it Perf}}


\newcommand{\fsigma}{{\sf F}_{\sigma}}
\newcommand{\gdelta}{{\sf G}_{\delta}}

%%%% Cardinal coefficients
\newcommand{\cof}{{\it cof}}
\newcommand{\cov}{{\it cov}}
\newcommand{\non}{{\it non}}

\newcommand{\cantor}{\ca}
\newcommand{\ucrz}{UCR_0}
\newcommand{\borelucrz}{Borel-UCR_0}
\newcommand{\Bor}{{\it Bor}}
%%%\newcommand{\proof}{\flushleft{ \sc Proof. } \\ }
%%%\newcommand{\restriction}{{\hbox{$\,|\grave{}\,$}}}
\newcommand{\szerod}{(s^2_0)}
\newcommand{\la}{\langle}
\newcommand{\ra}{\rangle}
\newcommand{\Part}{{\it Part}}
\newcommand{\homeomorphic}{\approx}
\newcommand{\Perf}{{\mathrm Perf}}
\newcommand{\calB}{{\cal B}}
\newcommand{\calF}{{\mathcal{F}}}
\newcommand{\cF}{\calF}
\newcommand{\calG}{{\cal G}}
\newcommand{\cG}{\calG}
\newcommand{\calI}{{\mathcal{I}}}
\newcommand{\calM}{{{\mathcal{M}}}
\newcommand{\meager}{\mathcal{M}}
\newcommand{\Even}{\hbox{\rm \tiny Even}}
\newcommand{\Odd}{\hbox{\rm \tiny Odd}}
\newcommand{\minideal}{${\cal F}_{\hbox{\rm \scriptsize min}}(\neg D)\;$}
\newcommand{\dom}{{\rm dom}}
\newcommand\Hom{{\rm Hom(\ca)}}
\newcommand{\cf}{{\rm cf}}

\newcommand{\gameonideal}{{\cal G}_\calI^{\it Tree}}
\newcommand{\treeproperty}{{P}^{\it Tree}}

%%%\newcommand{\blacksquare}{\begin{flushright} \rule{2mm}{2mm} \end{flushright}}


\def\volinfo{\textsc{Vol.} \textbf{\currentvolume}, No. \textbf{\currentissue}}
\def\publname{Mathematica Moravica}
\def\copyrightyear{2019}
\def\copyrightholder{Mathematica Moravica}
\def\currentvolume{23}
\def\currentmonth{}
\def\currentyear{2019}
\def\currentissue{2}
\pagespan{1}{12}

%%%%%%%%%%%%%%%%%%%%%%%%%%%%%%%%%%%%%%%%%%%%%%%%%%%%%%%%%%%%
%%%%%%%%%%%%%%%%%%%%%%%%%%%%%%%%%%%%%%%%%%%%%%%%%%%%%%%%%%%%



%Here authors can define environments and their numbering schemes.
%Authors should add environments that are to be used in the paper

\newtheorem{theorem}{Theorem}[section]
\newtheorem{proposition}{Proposition}[section]
\newtheorem{corollary}{Corollary}[section]
\newtheorem{lemma}{Lemma}[section]

\theoremstyle{definition}
\newtheorem{definition}{Definition}[section]
\newtheorem{remark}{Remark}[section]
\newtheorem{example}{Example}[section]

%%%%%%%%%%%%%%%%%%%%%%%%%%%%%%%%%%%%%%%%%%%%%%%%%%%%%%%%%%%%
%%%%%%%%%%%%%%%%%%%%%%%%%%%%%%%%%%%%%%%%%%%%%%%%%%%%%%%%%%%%
%Authors can also define custom functions or macros to make typing of frequently used complicated expressions easier.
%
\newcommand\dd{\mathbf{d}}     %to be used ONLY within math environment.
%\newcommand\im{\mathrm{Im}}    %Imaginary part using ordinary upright font
%\DeclareMathOperator{\real}{\mathrm{Re}}
%
%Another way to achieve the result similar to the previous case,
%but this one looks more natural when typeset (because of small space after it).
%One can define more complicated functions, as for example
%
%\newcommand{\binsq}[2]{\displaystyle{\genfrac{[}{]}{0pt}{}{#1}{#2}}}
%
%which looks like binary operator, but with square brackets instead of ordinary ones


\begin{document}

\long\def\symbolfootnote[#1]#2{\begingroup%
\def\thefootnote{\fnsymbol{footnote}}\footnote[#1]{#2}\endgroup}

%and use \symbolfootnote[1]{footnote} to get an *
%and use \symbolfootnote[0]{footnote} to get footnote without any symbol
%and use \symbolfootnote[2]{footnote} to get a dagger
%and use \symbolfootnote[3]{footnote} to get a double dagger

\title[On G - transitive version of always of the first category sets]{
On G - transitive version of always of the first category sets}

\author[Andrzej Nowik]{Author's Name$^{\ast}$}
%\ast is used as a reference sign for some information related to the paper,
%which will be put in the footnote of the first page (like research support information).
%If no such information is provided, \ast command can be removed.
\address{University of Gda\'nsk, \newline
Institute of Mathematics \newline
Wita Stwosza 57 \newline
80 -- 952 Gda\'nsk \newline
Poland}
\email{andrzej.nowik@ug.gda.pl}

%If there are more than one author, \author, \adress and \email commands should be repeated here with
%appropriate information

\date{}
\subjclass[2010]{Primary: 03E15; Secondary 03E20, 28E15.} %%% TODO: sprawdzic czy ams2010 
%%%%% classification jest nadal aktualne (bo w oryginalnym pliku bylo z ...2000 - sic!)
\keywords{Always first category sets, strongly meager sets,
WQN sets, $AFC^{\prime}$ sets.\\
\indent {\textit{Full paper}. Received 1 July 2019,
revised 1 September 2019, accepted 1 October 2019, 
available online 20 October 2019}}

\begin{abstract}
  We study the $G-$ invariant version of always of
the first category sets.
  This notion is a generalization of the notion of
$\afcp$ sets.
\end{abstract}

\maketitle

%%% DO NOT DELETE %%%
%%% field for doi number to be added %%%
%\begin{textblock}{4}[0,0](9.45,1)
%{\footnotesize doi: }
%\end{textblock}


\section{Introduction}
\symbolfootnote[0]{$^\ast$Here goes some paper related information, like Research support etc.}
%if there are no such information, this command can be removed or commented out.



%%%%%%%%!!!

\bigskip
\flushleft{\bf Definitions and notation.} \\
\bigskip

We consider the Cantor set $2^\omega$ 
which is a topological group (where 
$(x+y)(k) = x(k) + y(k) \mathit{mod} 2$).
%%%TODO: define 2^<omega!
For any $s\in 2^{<\omega}$ denote 
$[s] = \{x\in\cantor\colon s \subseteq x\}$.

Let $\Perf$ stand for the family of all perfect subsets of the space
$2^\omega$.
  Throughout the paper we assume that every $\sigma$ - ideal
$\calI$ is such that
  $\forall_{x\in X} \lbrace x \rbrace \in \calI$.

%%%%% zbedne, przynajmniej na razie
%  Let us recall that $X\seq\ca$ is \underline{strongly meager}
%(SFC) iff for every measure zero set $A\seq\ca$, there is $t\in\ca$,
%so that $(X + t) \cap A = \emptyset$.
%  A subset $X \subset \ca$ is a wQN-set if for each sequence of continuous
%functions
%$f_k:X \to \real$ with $f_k \to 0$ there is a subsequence $k_l$
%such that $f_{k_l} \stackrel{QN}{\to}0$ i.e. there exists a partition
%of $X$ into $\oo$ many sets $X_{n}$ ($n < \oo $), such that
%$f_{k_l} \restriction X_k \to 0$ uniformly as $l \to \infty$.

  By $Hom(X)$ we denote the group of all homeomorphisms of the
topological space X.
  We always assume that G is a fixed subgroup of $\Hom$.

The following additional terminology will be useful in our proof.

  For an arbitrary $g\in G$ and $Q \in \perf$
we often abbreviate the image $g(Q) = \lbrace gx: x\in Q \rbrace$
as simply $gQ$.

  We denote by $\mgr(P)$ the collection of all
first category sets on P, where $P \in Perf(X)$.

  We use a letter $\neglig$ to denote
the sigma ideal of Lebesgue measure zero sets of $2^\omega$.

  We denote by $\trans$ the subgroup of all
translations of $\cantor$. 

\bigskip
\flushleft{\bf Introduction.} \\
\bigskip

The aim of this paper is to investigate the $G-$ invariant
version of always of the first category sets.
First, recall the classical definition of always of the first category sets:
\begin{definition}
A set $X$ of $\cantor$ is an always of the first category ($\afc$) 
iff for every $P\in \Perf$, $X\cap P$ is a first category set
in the topology of $P$.
\end{definition}

The following notion of sets was first defined in \cite{NSW}
and then it has been studied most extensively in
papers \cite{NW1} and \cite{NW2}.

\begin{definition}
  A set $X$ of $\cantor$ is an $\afc'$--set if for each perfect
set $P$ there is an $\fsigma$--set $F$ such that $X\subseteq F$
and for each real $t$, $(F + t)\cap P$ is a first
category set in the topology of $P$.
\end{definition}

%%%%% zapis powtarzajacy poprzednia definicje, wiec zapewne zbedny
%\[
%\forall_{x \in \cantor} (P + x) \cap F \in \mgr(P + x).
%\]

It is known (see for example \cite{NSW}),
that $\afcp \seq \afc$ and every Sierpi\'nski set is an $\afcp$ set.
Recall that $S \subseteq \cantor$ is a {\it Sierpi\'nski set}
if, and only if, it is uncountable and has countable intersection with
any set of measure zero. Recall also that under the assumption
of Continuum Hypothesis there exists a Sierpi\'nski set (see \cite{Si}).

Let us define the main notion of this article.

\bigskip

{\bf The $\afcg$ - sets}

%%%Let us start with the following definition.

Suppose that G is a subgroup of $\Hom$
and let $X$ be an arbitrary subset of $\ca$.

\begin{definition}
\label{afcg}
We say that $X \in \afcg$ iff
for every $Q \in \perf$ there
exists $F \supseteq X$, $F\in \fsigma$
such that
$\forall_{g\in G} gQ \cap F \in \mgr(gQ)$.
\end{definition}

  This notion is a natural generalization
of the notion of $\afcp$ sets.

\flushleft{\bf Remarks:} \\

  It is obvious that $\afcp_{\trans} = \afcp$,
$\afcp_{\lbrace id \rbrace} = \afc$
and $\afcp_{\Hom} = [\cantor]^{\leq \omega}$.
  It is also evident that if $G_1 \seq G_2$ then
$\afcp_{G_1} \supseteq \afcp_{G_2}$.
  Summarizing this we obtain
the following chart, in which arrows
$\longrightarrow$ denote inclusions.

\begin{center}
\setlength{\unitlength}{1mm}

\begin{picture}(140,50)(5,0)

\put(25,40){$\afcp_{\Hom}$}
\put(60,40){$\afcp_{\trans}$}
\put(100,40){$\afcp_{\lbrace id \rbrace}$}

\put(25,20){$[\ca]^{\leq\oo}$}
\put(60,20){$\afcp$}
\put(100,20){$\afc$}

\put(45,42){\vector(1,0){15}}
\put(80,42){\vector(1,0){15}}

\put(27,25){\line(0,1){13}}
\put(28,25){\line(0,1){13}}

\put(62,25){\line(0,1){13}}
\put(63,25){\line(0,1){13}}

\put(102,25){\line(0,1){13}}
\put(103,25){\line(0,1){13}}

\end{picture}
\end{center}

%%%%%%%%%%%%%%%%%%%%%%%%%%%%%%%%%%%%%%%%%%%%%%%%%%%%%%%%%%%%%%%%%%%%%%%%%%%%
%                                                                          %
%                                                                          %
%                     Wlasnosc (Em)                                        %
%                                                                          %
%                                                                          %
%%%%%%%%%%%%%%%%%%%%%%%%%%%%%%%%%%%%%%%%%%%%%%%%%%%%%%%%%%%%%%%%%%%%%%%%%%%%

The following definition will be useful for the problems in this section.

\begin{definition}
\label{def_em}
Let $\calI$ be a $\sigma$ - ideal of subsets of the space $\ca$.

We say that a group $G \leq \Hom$ {\it has the $(Em)_{\calI}$ property}\/
iff there exists a perfect set $Q \in \perf$
such that for each
$P\in \perf \setminus \calI$
there exists $g \in G$ such that
$P \cap gQ \not\in \mgr(gQ)$.
\end{definition}

\flushleft{\bf Remarks:} \\

One can prove that $\trans$ does not have the $(Em)_{\neglig}$ property.

Without loss we may assume that in Definition \ref{def_em}
$P$ is only closed set such that $P \not\in \calI$.

We will start with the following theorem.

\begin{theorem}
\label{em=>afcg->i}
  Let $\calI$ be an arbitrary $\sigma$ - ideal of subsets of $\ca$
such that $\forall_{x\in\ca} \lbrace x \rbrace \in \calI$.

  Moreover, let $G \leq \Hom$
be a subgroup of $\Hom$ with the property $(Em)_{\calI}$.

  Then we have:
$\afcg \seq \calI$.
\end{theorem}

\proof

Let $X \seq \ca$ be a set such that $X \not\in \calI$.
From the definition of
the notion $(Em)_{\calI}$ we get a perfect set $Q$ such that
for each closed $E \not\in \calI$ we have
$\exists_{g \in G} E \cap gQ \not\in \mgr(gQ)$.

Let $F \seq \ca$ be an $F_{\sigma}$ set
such that $X \seq F$.
We have that
$$F = \bigcup_{ \nlto } F_n$$
where $cl(F_n) = F_n$ , so there exists $n_0 < \omega$
such that $F_{n_0} \not\in \calI$.
Now there exists $g \in G$
such that $F_{n_0} \cap gQ$
is not meager in gQ.
So we conclude, that X is not an $\afcg$ set.

\bigskip

The implication given in Theorem \ref{em=>afcg->i} is reversible
under some additional set theoretical assumptions.
Indeed, we have the following theorem:

\begin{theorem}
\label{em<=>afcg->i}
  Let us assume like in Theorem \ref{em=>afcg->i}
that $\calI$ is an arbitrary $\sigma$ - ideal of subsets of $\ca$
such that $\forall_{x\in\ca} \lbrace x \rbrace \in \calI$ and
$G \leq \Hom$ is a subgroup of $\Hom$.
  Moreover, assume that
%%%Another possible way to construct a counterexample
%%%is the consideration of the $\sigma$ -- ideal of $s_0$ sets.
\begin{enumerate}
  \item
    \[ \cof(\calI) = \cov(\calI) \leq \non(\afcg), \]
  \item
    \[ \forall_{P\in\perf\setminus\calI} \exists_{|C| \leq \oo}
      \ca \setminus (P + C) \in \calI, \]
  \item
    \[ \trans \seq G. \]
\end{enumerate}

Then the following conditions are equivalent:

\begin{enumerate}
  \item
    $\afcg \seq \calI$
  \item
    $G$ fulfills $(Em)_{\calI}$.
\end{enumerate}
\end{theorem}

\proof

Theorem \ref{em=>afcg->i}
gives us immediately the implication $(2) \Rightarrow (1)$.


  Now suppose that $G$ fulfills $\neg (Em)_{\calI}$.
Since $\kappa = \cof(\calI) = \cov(\calI)$
and $\calI$ contains singletons we conclude that
there exists a $\kappa$ - Sierpi\'nski set $X$ with
respect to $\calI$, i.e. a set $X$ of size
$\kappa$ such that
  \[
    \forall_{A \in \calI} |A \cap X| < \kappa.
  \]



Let $Q \in \perf$ be arbitrary.
From our assumption $\neg (Em)_{\calI}$
there exists a perfect set P
such that $P \not\in \calI$ and
$\forall_{g \in G} gQ \cap P \in \mgr(gQ)$.
  Pick a countable set $C \seq \ca$ such that
$\ca \setminus (C+P) \in \calI$.

We have

$X =
  \big[ [\ca \setminus (P+C) ] \cap X \big]
\cup
  \big[ (P+C) \cap X \big]$.

Since $\ca \setminus (P + C) \in \calI$ we obtain

\[ | [ \ca \setminus (P + C) ] \cap X | < \kappa. \]

Moreover, if $c \in C$ and $g \in G$, then
$hQ \cap P \in \mgr(hQ)$,
where $h \in G$ is defined by $h(x) = g(x) - c$.
  Hence $gQ \cap (P + c) \in \mgr(gQ)$,
thus
  $gQ \cap (P + C) \in \mgr(gQ)$ for each $g \in G$.

  Since $\kappa \leq \non(\afcg)$ we obtain
$[ \ca \setminus (P+C)] \cap X \in \afcg$,
so there exists $E \in \fsigma$,
$E \supseteq X \setminus (P + C)$ such that
$\forall_{g \in G} gQ \cap E \in \mgr(gQ)$.
  Finally, define $E^* = E \cup (P + C)$.
It is easy to see that $X \seq E^*$ and
$\forall_{g \in G} gQ \cap E^* \in \mgr(gQ)$.
  Hence $X \in \afcg$ and the proof is
completed, since X does not belong to $\calI$.
\medskip

  Unfortunately, we don't know whether this
theorem is true under weaker assumptions.
  Thus we think that the following question may be of
some interest.

\begin{question}
Can we prove the equivalence given in the previous theorem
under weaker set theoretical assumptions?
\end{question}

%%%%%%%%%%%%%%%%%%%%%%%%%%%%%%%%%%%%%%%%%%%%%%%%%%%%%%%%%%%%%%%%%%%%%%%%%%%%
%                                                                          %
%                                                                          %
%             Characterization (Em)                                        %
%                                                                          %
%                                                                          %
%%%%%%%%%%%%%%%%%%%%%%%%%%%%%%%%%%%%%%%%%%%%%%%%%%%%%%%%%%%%%%%%%%%%%%%%%%%%

  For any $\cF \subseteq \Perf$ 
let us define the following useful cardinal coefficient:
\begin{definition}
$Em(\cF, G) = \min\{|\cG|\colon \cG\subseteq \Perf \wedge 
\forall_{P\in\cF} \exists{g\in G} \exists_{Q\in\cG} gQ \subseteq P\}$
\end{definition}

  Let us formulate a characterization of the property 
$(Em)$ in terms of previously defined coefficient.

Assume that $G$ has the property that for each $x\in\cantor$
the orbit $Gx$ is dense in $\cantor$.
Then the following conditions are equivalent:

\begin{enumerate}
\item
  $G$ fulfills $(Em)_{\calI}$; 
\item
  $|Em(\Perf\setminus \calI, G) | \leq \aleph_0$.
\end{enumerate}

  We will need the following technical lemma
(folklore for the group $G = \trans$):
\begin{lemma}\label{lemma-dense}
If $G\leq Hom(\cantor)$ is a group such that for each 
$x\in\cantor$, $Gx$ is dense in $\cantor$,
then for every sequence $\langle Q_n\rangle$
of perfect subsets of $\cantor$ there exists
a perfect $P\in\Perf$ such that 
$\forall_{n\in\omega} \exists_{g\in G} gQ_n \cap P 
\not \in \meager(P)$.
\end{lemma}
\begin{proof}
Let $v_k = [(0,\ldots,0, 1)]$ ($0$ $k$ times).
For each $k$ choose $x_k\in Q_k$ and $g_k\in G$
such that $g_k x_k \in V_k$. Define
$P = \overline{\bigcup_{k\in\omega}g_k Q_k \cap V_k}$,
then $P$ is a perfect set and if $k\in\omega$
then 
$g_k Q_k \cap P \supseteq g_k Q_k \cap V_k \not\in \meager(P)$.
\end{proof}

\begin{proof}%%of Theorem
$(1)\to (2)$.
Assume that $G$ has the $(Em)_\calI$ property,
i.e. there exists $Q\in\Perf$ such that
$\forall_{P\in\Perf\setminus\calI}\exists_{g\in G} 
P\cap gQ\not\in \meager(gQ)$. 
Let us define perfect sets:
$\cG = \{Q \cap [s] \colon Q \cap [s] \not= \emptyset \wedge
s\in 2^{<\omega}\}$.
Then $|\cG| \leq \aleph_0$ and if $P\in\Perf\setminus\calI$
then there exists $g\in G$ such that
$P\cap gQ \not\in\meager(gQ)$, so
$P\cap gQ \supseteq W\cap gQ \not= \emptyset$
for some open set $W$.
Then $g^{-1}[W] \cap Q \not\emptyset$
so there exists $Q_1\in \cG$ such that
$Q_1 \subseteq g^{-1}[W] \cap Q$.
Hence $g[Q_1] \subseteq W \cap g[Q] \subseteq P \cap g[Q]$.
This proves (2).

$(2)\to(1)$.

\end{proof}

 
  Below we give an useful characterization of the
property $(Em)_{\neglig}$.

\medskip

\begin{theorem}
Let G be a subgroup of $\Hom$ which contains the
subgroup $\trans$. The following two conditions are
equivalent:

\[ (1)\ \ \neg (Em)_{\neglig}, \]
%%%i.e.
%%%$\forall_{Q \in \perf} \exists_{P \in \perf \atop \mu(P) > 0}
%%%\forall_{g \in G}
%%%P \cap gQ \in \mgr(gQ) $

\par (2) For every $Q \in \perf$
and for every $\eps > 0$ there exists an open
set U, such that
$\mu(U) < \eps$
and $\forall_{g \in G} gQ \cap U \not = \emptyset$
\end{theorem}

\proof

$(1) \Rightarrow (2)$
\par
Assume that
$\forall_{Q \in \perf} \exists_{P \in \perf \atop \mu(P) > 0}
\forall_{g \in G} gQ \cap P \in \mgr(gQ)$

Let $Q \in \perf$ be any perfect set and let $\eps > 0$.
Pick a perfect set $P$, $\mu(P) > 0$ such
that $\forall_{g \in G} gQ \cap P \in \mgr(gQ)$.
We can find finite $C \seq \cantor$ such that
$\mu(\cantor \setminus (C+P)) < \eps$.\
Now put $U = \cantor \setminus (C+P)$.

By way of contradiction suppose that
there exists $g \in G$ such that
$gQ \cap U = \emptyset$.
Then $gQ \seq C + P$, hence
there exists $c_0 \in C$ and an open set $I$
such that $\emptyset \not = I \cap gQ \seq P + c_0$.
  Define $h(x) = g(x) - c_0$, obviously $h \in G$.
Next, $hQ = gQ - c_0$ thus
$\emptyset \not = hQ \cap (I - c_0) \seq P$,
which is a contradiction with
$hQ \cap P \in \mgr(hQ)$.

%%%so if $g_0$ denotes the translation
%%%$x \rightarrow x-c_0$
%%%there would be $g_0(Q) \cap g_0(I) \seq P$
%%%5a contradiction, because
%%%$g_0(Q) \in \ooo$

\medskip

$(2) \Rightarrow (1)$

Assume (2). Let $R$ be any perfect
set. Let $\{ I_m \} _{m < \omega}$
be an enumeration of all basic clopen sets of $\cantor$.
Let
  \[\eps_{m} = {1 \over 2^{m + 2}}.\]
For any $m < \omega$
we choose, using the assumption (2),
an open set $U_{m}$ such that

\[
\forall_{g\in G} R \cap I_m \not = \emptyset
\Rightarrow U_{m} \cap g(R \cap I_m) \not = \emptyset
\]

and $\mu(U_{m}) < \eps_{m}$.
This can be done, since
$I_m \cap R$ is a perfect or an empty set.

Now put

\[ U = \bigcup_{m < \omega} U_m. \]

We see that
\[
\mu(U) \leq \sum_{m < \omega}
{1 \over 2^{m + 2}} \leq 2 \cdot {1 \over 4} < 1.
\]

Define $F = \cantor \setminus U$, then we have
$\mu(F) > 0$
so choose a perfect $P \seq F$ of positive
measure.

  Let $g \in G$ and $I_{m_0}$ be given such that
$R \cap I_{m_0} \not = \emptyset$.

Now $U_{m_0} \cap g(R \cap I_{m_0}) \not = \emptyset$.
Moreover, since
$U_{m_0} \cap P = \emptyset$
we obtain that $g(R \cap I_{m_0}) \not \seq P$.
This means that (1) holds.

  \[ \square \]

Notice that in the proof of implication (2)$\Rightarrow$(1)
we did not use the assumption that
$\trans \leq G$.

\bigskip

  In the next part we will prove theorems about relations between
$\afcg$ and different classes of peculiar small sets of the real line.

\begin{theorem}
\label{-em=>N}
  Assume that $G$ is a subgroup
of $\Hom$ which contains $\trans$.
  If $G$ fulfills $\neg (Em)_{\neglig}$,
then every strongly meager set is an $\afcg$ set.
\end{theorem}

\proof

Let X be a strongly meager set.
Let $Q$ be an arbitrary perfect set.
From $\neg (Em)_{\neglig}$ we know that there exists a perfect
set $P$ such that $\mu (P) > 0$ and
  \[ \forall_{g \in G} g(Q) \cap P \in \mgr(g(Q)). \]
Take now countable $C \seq \cantor $ such that
$\cantor \setminus (P + C) \in \neglig$.
  Then there exists $x_0$ such that $(x_0 +X) \cap
[ \cantor \setminus (P + C) ] = \emptyset $ ,
so $x_0 + X \seq P + C$, hence $X \seq P + C - x_0$.
  Let $g \in G$ be an arbitrary and let $c \in C$.
Define $h \in G$ by $h(x) = g(x) - c + x_0$.
  Then $h(Q) \cap P \in \mgr(h(Q))$, hence
$\big( g(Q) - c + x_0\big) \cap P \in \mgr(g(Q) - c + x_0)$,
thus $g(Q) \cap (P + c - x_0) \in \mgr(g(Q))$.
  Since $c \in C$ was taken arbitrary, we conclude that
$g(Q) \cap (P + C - x_0) \in \mgr(g(Q))$.
  This implies that $X \in \afcg$, since $P + C - x_0 \in \fsigma$.

%%%Take $g \in G$ such that
%%%$g(x) = X + x_0$. Now let $Q \in \ooo$ be any perfect set from
%%%$\ooo$. We have now, that
%%%$$g(X) \in \mgr(Q)$$
%%%Because $Q\in \ooo$ was taken arbitrary, this leads us
%%%to conclude, that $\forall_{ Q \in \ooo} X \in \mgr(Q)$

$$\square$$

\flushleft{\bf Remark:} \\

This implication is reversible
under CH. Namely:

\begin{theorem}
Suppose that $G \leq \Hom$ and assume that $G$ has the $(Em)_{\neglig}$
property. Moreover, assume CH. Then there exists a strongly
meager set $X\seq\ca$ such that $X \not\in \afcg$.
\end{theorem}


\proof

  Let $X\seq\ca$ be arbitrary Sierpi\'nski set.
It is well known (see \cite{P}) that $X$ is
strongly meager.
  From the $(Em)_{\neglig}$ property we obtain that
there exists $Q \in \perf$ such that
  \[
    \forall_{P \in \perf \setminus \neglig}
    \exists_{g \in G} P \cap g(Q) \not\in \mgr(gQ).
  \]
Suppose that $E$ is an $\fsigma$--set such that $X \seq E$.
Since $X \not \in \neglig$ it follows that $E \not \in \neglig$.
  Hence there exists $P \in \perf \setminus \neglig $ such that
$P \seq E$

Therefore $\exists_{g \in G} P \cap g(Q) \not\in \mgr(gQ)$,
hence $E \cap g(Q) \not\in \mgr(g(Q))$.
  This yields $X \not\in \afcg$, which finishes the proof.

\begin{corollary}
\label{corollary_1}
  Assume that $\cov(\neglig) = \cof(\neglig)$ and
$\cov(\neglig)$ is a regular cardinal.
  Let $G \leq \Hom$ and suppose that $\trans \leq G$.
Then the following conditions are equivalent:

  \begin{enumerate}
  \item
    $G$ has the $(Em)_{\neglig}$ property.

  \item
    $\afcg \seq \neglig$.

  \end{enumerate}
\end{corollary}

\proof
  The implication $(1) \Rightarrow (2)$ follows immediately
from Theorem \ref{em=>afcg->i}.
  Assume $\neg (Em)_{\neglig}$. Since $\cov(\neglig) = \cof(\neglig)$,
there exists a $\cof(\neglig)$ -- Sierpi\'nski set.
  By Lemma 8.5.4 from \cite{BJ}
if there exists a $\kappa$ -- Sierpi\'nski set
and $\cf(\kappa) = \kappa > \oo$,
then every set of size $< \kappa$ is strongly meager.
  Hence by Theorem \ref{-em=>N} we conclude that
$\non(\afcg) \geq \cof(\neglig)$
%%%  Next, it is sufficient to observe that in this case
%%%we have
%%%  $\cov(\neglig) = \non(\mgr) = \cof(\mgr) = \cof(\neglig)$,
thus all assumptions of Theorem \ref{em<=>afcg->i} are satisfied.
  $\square$

%%%%%%%%!!!



%References should be typeset within thebibliography environment
\begin{thebibliography}{99}  % use two-digit argument if there are 10 or more references, one-digit otherwise
\setlength{\itemsep}{6pt}

\bibitem{1} Author 1, \textit{Book Title}, Publisher, 2016.

\bibitem{2} Author 2, Author 3, Author, 4, \textit{Title 2}, Journal name, 11(2) (2016), 22-31.

\bibitem{3} Author 5, Author 6, \textit{Title 3}, Proceedings of the Conference, place, July 22-26 2018, 12-18.
%%%%%% tutaj wkopiowane z oryginalnego pliku:

%%%%%%%%%%%%%%%%%%%%%%%%%%%% The bibliography %%%%%%%%%%%%%%%%%%%%%%%%%%%%%%%%

\bibitem{BJ} T. Bartoszy\'nski, H. Judah, \textit{Set Theory: on the
strucure of the real line}, A. K. Peters, Wellesley, Mass., 1995.

\bibitem[N]{N}
A. Nowik
{\em Remarks about transitive version of perfectly
meager sets,}
{\bf Real Analysis Exchange} 22(1) (1996/97) 406 - 412.

\bibitem[NSW]{NSW}
A. Nowik, M. Scheepers, T. Weiss.
{\em The algebraic sum of sets of
    real numbers with strong measure zero sets,}
{\bf Journal of Symbolic Logic} vol. 63(1), 1998, 301 - 324.

\bibitem[NW1]{NW1}
A. Nowik, T. Weiss
{\em Not every Q - set is perfectly meager in the transitive sense.}
{\bf Proceedings of The American Mathematical Society} 128(496), No 10
(October 2000), 3017 - 3024.

\bibitem[NW2]{NW2}
A. Nowik, T. Weiss
{\em Not every Q - set is perfectly meager in the transitive sense.}
{\bf Proceedings of The American Mathematical Society} 128(496), No 10
(October 2000), 3017 - 3024.

\bibitem[P]{P}
J. Pawlikowski, {\em All Sierpi\'nski sets are strongly
meager}, 1992, {\bf Arch. Mat. Logic} 35 (1996) 281 -- 285.

\bibitem[Si]{Si}
W. Sierpi\'nski, {\em Sur l'hypoth\`{e}se du continu 
($2^{\aleph_0} = \aleph_1$)}, {\bf Fundamenta Mathematicae}, 5 (1)
(1924), 177–187
\end{thebibliography}

\end{document}

%%%%%% usuniety material:
%%% Zdecydowalem sie material o wlasnosci (Sp) usunac z tej wersji bowiem
%%% byl zbyt oczywistym uogolnieniem dosc specjalistycznej pracy
%%% o relacji miedzy wQN a AFC^'.

%%%%%%%%%%%%%%%%%%%%%%%%%%%%%%%%%%%%%%%%%%%%%%%%%%%%%%%%%%%%%%%%%%%%%%%%%%%%
%                                                                          %
%                                                                          %
%                     Wlasnosc (Sp)                                        %
%                                                                          %
%                                                                          %
%%%%%%%%%%%%%%%%%%%%%%%%%%%%%%%%%%%%%%%%%%%%%%%%%%%%%%%%%%%%%%%%%%%%%%%%%%%%

\begin{definition}
Let G be any subgroup of $\Hom$. We say that G has the
{\it (Sp) property}\/
iff for every perfect set $Q$
there exists a sequence $\{ U_i \} _{i < \omega} $
of clopen disjoint subsets of $\cantor$ such that
\[ \forall_{g \in G \atop i < \omega}
   U_i \cap g(Q) \not = \emptyset \]
\end{definition}

\bigskip

The following proposition is the $G - transitive$ analog
of the theorem that every $wQN$ set has the $\afcp$
property (see \cite{N}, Conclusion 2).

\begin{theorem}
Let G be a subgroup of $\Hom$. If G has the (Sp)
property, then every wQN set is an $\afcg$.
\end{theorem}

\proof

  Let G be a subgroup of $\Hom$ which fulfills (Sp).
Let X be a wQN set. Let $Q_0 \in \perf$
be an arbitrary perfect set.
  Let $\{ B_n \}_{n<\omega}$ be an enumeration of all
basic clopen subsets of $\ca$ such that
$B_n \cap Q_0 \not = \emptyset$.

Since $B_n \cap Q_0$ is a perfect subset of $\ca$ we can find
a sequence
$\lbrace U^{(n)}_i \rbrace _{i< \omega}$
of disjoint clopen subsets such that

\[
\forall_{g \in G \atop i < \omega}
g(B_n \cap Q_0) \cap U_i^{(n)} \not = \emptyset
\]

Next, define

\[
F_i(x) = \sum_{\nlto} {1 \over 2^n} \cdot \chi_{U^{(n)}_i} (x),
\]

where $\chi_A$ denotes as usually the characteristic function of the set A.

Next, it is easy to see that for each $i\in\oo$ $F_i(x)$ is
a continuous function and
$\lim_{i \rightarrow \infty} F_i(x) = 0$
pointwise.
  Thus from the definition of wQN sets we can find a subsequence
$F_{i_j}$ and a partition of X:
$X = \bigcup_{k<\omega} X_k $
such that $$F_{i_j} \restriction \overline{ X_k} \rightarrow 0$$
uniformly as $j \rightarrow \infty$.
  We put
  \[ E = \bigcup_{k<\omega} \overline{X_k} \]
Obviously, E is an $\fsigma$ set.

Take any $g \in G$.
By way of contradiction assume that
$E \cap g(Q_0) \not\in \mgr(g(Q_0))$.
Then $\overline{X_k} \cap g(Q_0) \not\in \mgr(g(Q_0))$
for some $k \in \oo$ and therefore
$g(B_{n_0} \cap Q_0) \seq \overline{X_k}$
for some $n_0 \in \oo$.
  Next for each $i \in \oo$,
$g(B_{n_0} \cap Q_0) \cap U^{(n_0)}_i \not= \emptyset$
and this is a contradiction since the sequence
$\lbrace F_{i_j} \rbrace_{j \in \oo}$ of functions converge
uniformly to zero on $\overline{X_k}$.

%%%It follows that
%%%\[ F \cap g(Q_0) \in \mgr(gQ_0) \]
%%%since otherwise we would have
%%%  \[ \overline{X_k} \cap g(Q_0) \not\in \mgr(gQ_0) \]
%%%for some $k\in\oo$,
%%%then we would have
%%%$B_{n_0}$ such that
%%%  \[ g(B_{n_0} \cap Q_0) \seq \overline{X_k} \]
%%%for some $n_0 \in \oo$.
%%%Now for every $i < \omega$ $g(B_{n_0} \cap Q_0) \cap U^{n_0}_i \not =
%%%\emptyset$
%%%so $F_{i_j}$ on $g(B_{n_0} \cap Q_0)$
%%%does not converge uniformly to zero, so

\smallskip
  \[ \square \]


\bigskip
\bigskip

  It is an easy observation that if G has the (Sp) property
then $G$ does not have the $(Em)_{\neglig}$ property.
Indeed, assume (Sp), so
\[ \forall_{Q \in \perf} \exists_{U_i\ disjoint\ \atop clopen }

\forall_{g \in G \atop i < \omega} U_i \cap gQ \not = \emptyset \]

 Let $\epsilon > 0$.
We can easy find $U_i$ such that $\mu (U_i) < \epsilon$.
Now
\[ \forall_{g \in G} gQ \cap U_i \not = \emptyset. \]

It follows immediately from Theorem 2 that
$G$ fulfills $\neg (Em)_{\neglig}$.

